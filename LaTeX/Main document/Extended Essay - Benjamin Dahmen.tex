\documentclass[]{scrartcl}

%PACKAGES
\usepackage[style=authoryear, backend=biber]{biblatex}
\usepackage[utf8]{inputenc}
\usepackage{amsmath}

%LIBRARY SETUP
\bibliography{../../../../../Hobbys-BenDahmen-PC/LaTeX/Libraries/Extended_Essay_LSE}

\title{The effects of the German minimum wage on youth employment and labour supply}
\subtitle{Extended Essay for the MSc Economics | London School of Economics}
\author{Benjamin Alessandro Dahmen}
\date{}
\begin{document}

\maketitle

\begin{abstract}
Abstract
\end{abstract}

\section{Introduction}
Minimum wages [MWs] have been subject to debates for much of recent history and are still nowadays. 92\% of the member states of the International Labour Organization (ILO) currently have a minimum wage \parencite{ILO_MW}.
\section{The German Minimum Wage Law}
The German Minimum Wage Law came into effect on 1\textsuperscript{st} January 2015. It initially set the federal MW level at 8.50€ and established that this level was to re-evaluated every two years. Such re-evaluations have led to further increases to 8.84€ in 2017, 9.19€ in 2019 and 9.35€ in 2020. While there was no universal MW prior to 2015, there existed various sector-specific MW agreements that were bargained between employer associations and employee unions. Sectors in which such an agreement existed that featured a wage lower than the new federal minimum were granted a transitory time period until 1\textsuperscript{st} January 2017 to match the federal MW. Moreover, the law permitted various types of exemptions for minors, trainees, interns, volunteers and the long-term unemployed. Factoring in these circumstances, the German Federal Statistical Office estimates that roughly 10.7\% of workers were initially affected by the new MW. Two groups that were particularly affected are females and the marginally employed. Marginal employment in Germany is defined as earning up to 450€ per month which is not subject to social security contributions.


\printbibliography

\end{document}
