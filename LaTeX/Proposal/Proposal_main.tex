\documentclass[11pt]{scrartcl}
\usepackage[utf8]{inputenc}
\usepackage{amsmath}
\usepackage[margin=6em]{geometry}
\usepackage[style=authoryear]{biblatex}
\addbibresource{"C:/Users/benda/OneDrive/Documents/Hobbys-BenDahmen-PC/LaTeX/Libraries/Extended_Essay_LSE.bib"}

\title{Extended Essay Research Proposal}
\subtitle{EC426 Public Economics}
\author{Benjamin Alessandro Dahmen}

\setlength{\parindent}{0pt}

\begin{document}
	\maketitle
	
\section{Introduction}
One of the main objectives of governments is to create well functioning labor markets that both guarantee a high level of employment and at the same time ensure fair working conditions for all participants. A very common tool that focuses especially on the latter objective is the minimum wage. Especially in Europe, but also elsewhere, they have been used extensively recently. However, the academic debate regarding the exact employment effects of minimum wages [MW] is still very active and ambivalent. While a large part of the literature suggests that the overall employment effects of MW are at worst moderate, there is also some evidence suggesting otherwise. A particularly interesting issue is the effect on the employment of young workers. In many countries the MW regulations include special rules and exemptions for workers below certain ages. This is often justified by the notion that young workers face a very elastic labor demand and therefore would be hurt by a MW. Given these circumstances, the effect on young workers is an especially relevant topic to be investigated. \\ \\
In my Extended Essay I would like to look at this effect in the context of the introduction of the German MW law ("Mindestlohngesetz") in 2015. The law was meant to unify the dispersed MW structure prior to that date. Before 2015 most wage floors were set by collective bargaining agreements from different unions. Each of these would cover individuals employed in different sectors, while many sectors remained entirely without a wage floor. Furthermore, the German government put emphasis on ensuring that the MW holds strictly for any individual who is at least 18 years old. This results in a very strong discontinuous age cutoff in the wages that need to be payed to young workers. In my analysis I want to use a regression discontinuity [RD] design to estimate the employment effects of the MW introduction for individuals who are close to 18 years old. While the abrupt nature of this discontinuity provides a great context to observe employment effects it also raises some doubts about the identifying assumptions of an RD. There are many reasons to believe that individuals one day before and after their respective 18th birthdays differ significantly. Turning 18 is arguably one of the most impactful events in one's life as many rules change from one day to the other. Therefore, it will be crucial to show that the smooth development of employment is robust to the many other changes at age 18.

\section{Previous Literature}
There exists a very broad literature around the employment impacts of MW and a lot of studies pay attention to young or low-wage workers specifically. Most empirical evidence in the late 90s and early 2000s has come from Card as well Neumark and Wascher. \textcite{CardMWVariation1992} looks at the April 1990 increase in the US federal MW level. He uses the fact that the share of workers whose initial wages lies in between the new and old MW levels vary per state. This is mainly the case as prior to the raise many states had set their own individual MW levels. This variation is used to estimate employment effect for 16-19-year-olds. Card finds a positive effect on the average wage earned but, contrary to predictions by the theory, no effect on employment levels. In another study Card (\citeyear{CardMW1992}) looks at MW increases by individual US states only. He picks California as a particularly interesting case as their increase is one of the largest and, furthermore, California has a high share of affected workers. Also in this analysis Card focuses on teenage workers as well as retail trade employees as that industry has a high share of MW employment in California. Again he is unable to confirm the theoretical prediction of adverse employment effects, but quite to the contrary, finds positive effects on both wage levels and employment. \\

\textcite{NeumarkWascherMW1992} use panel data on the evolution of the individual US state MW rates over time. Further, they also look at the introduction of so called "subminimum wages" that apply for new workers, such as young adults entering the labor force. These lower MW were supposed to make it easier for new workers to set foot in the labor market. In their analysis they find very small negative effects on employment and also evidence that the subminimum wages moderated the employment effects for young workers. In later research (\cite{NeumarkWascherMW2001}) they repeat their analysis with improved data and pay additional attention to how MW increases, affect schooling and on-the-job-training opportunities for employees. Their analysis reveals that MW increases can lead to less training offerings from employers and that this effect is not balanced by increased training opportunities for unemployed workers. \textcite{PereiraMWPortugal2003} looks at a MW increase in Portugal that applies to young workers only. Specifically, the government raised the youth MW from 75\% to 100\% of the already existing adult MW level. He argues that this setting is especially relevant as the average wage in Portugal is much closer to the MW than in other developed countries and, moreover, Portugal has a large secondary sector that is more exposed to MW changes through international competition. Pereira finds only moderate employment decreases, although they are larger than in most of the previous literature. Furthermore, he provides evidence of firms substituting towards older workers who have become relatively cheaper.\\

Although much of these early studies have found only limited employment responses, some subsequent analyses have estimated more diverse results. \textcite{StrainMW2017} conduct a preliminary analysis of US MW increases in the post-financial crisis era. They use a triple differences design to estimate the effect of individual states raising their levels while other states don't. Contrary to much of the previously presented literature, they find that MW increases slow down the growth of employment rates, particularly for young workers. Next to that they report that actively decided increases have stronger effects than automatic inflation-indexed increases. Jardim et al. (\citeyear{JardimMWSeattle2017}) look at a more geographically confined context in which Seattle increased its MW twice in the span of a few years. The authors have access to administrative data which enables them to compare a wide range of approaches that have been used in previous literature. In their main analysis they find no effects of the MW raise to 11\$ but significant disemployment effects for the subsequent raise to 13\$. Looking more closely at the labor market responses it also seems that intensive margin changes could be more important than changes on the extensive margin, especially when much of the low-wage work is of part-time nature. Lastly, the authors compare their results to using a differences-in-differences [DID] approach to compare regions in Seattle that are close to the border to unaffected regions that are adjacent. In this second analysis they report that the parallel trends assumption is violated, possibly because neighboring regions pick up some of the labor market changes even if their MW don't change. This result may cast some doubt on earlier DID approaches that don't show convincing parallel trends. \\

Lastly, I summarize two recent papers that are the most close in approach to my own idea as they look at age dependent cutoffs for young workers specifically. \textcite{KabatekBirthday2020} looks at the particular setup of the Dutch MW for young workers. In the Netherlands there is a step-wise increase in MW from age 16 to 23, resulting in a difference of 300 percent between the lowest and highest level. Kabatek finds that after an individual's birthday they are more likely to experience a job separation, possibly because they have become more expensive to employers. However, in the medium term, he also finds that following a birthday there is an increased chance of a new job creation. This could imply that also the supply adapts to the MW step function and that young workers wait to be eligible for a higher wage before applying to a job. In another paper Kreiner et al. (\citeyear{KreinerReckMW2020}) look at the MW setting in Denmark where the full MW applies to anyone turning 18 but not at all before that. The authors apply a RD design to this very abrupt cutoff and compare individuals slightly before and after their 18\textsuperscript{th} birthdays. They find an elasticity of -1 which implies relatively strong disemployment effects and no effects on overall wage income. Furthermore, they see that loosing one's job at 18 has strong long-term effects as such individuals are much less likely to have a job two years later than individuals who did not lose their job at 18. Contrary to Jardim et al. (\citeyear{JardimMWSeattle2017}) this study finds that most disemployment is concentrated along the extensive rather than intensive margin. \\

\subsection*{My contribution}
I plan to contribute to this rather extensive literature by specifically looking at cutoffs for very young workers, as this topic has not been studied as extensively yet. The main literature for this approach is the above mentioned Kreiner et al. (\citeyear{KreinerReckMW2020}) paper. Although their research design is convincing in many aspects, some doubts can be raised regarding the smooth development of other variables influencing employment around the 18 year age cutoff. I want to address this problem by comparing the employment change before and after the implementation of the MW law. This helps me to control for changes to employment at age 18 that happen even in the absence of a MW. Moreover, I will also add to the literature looking at the German minimum wage law specifically. Given how recently the law was implemented, its effects have so far been analyzed by a few papers only. These papers mostly focus on the effects on employees working in industries with a high share of low-wage jobs, which means that I will analyze the law from a novel perspective.
\section{Data and Preliminary Empirical Strategy}
For my analysis I plan on using data from the German Socio-Economic Panel [SOEP]. As I plan to use this proposal to apply for the SOEP data set, much of the empirical strategy is based on descriptive information about the SOEP and, therefore, preliminary. The SOEP consists of roughly 15.000 households and more than 25.000 respondents (\cite{DIWWebsite}). Based on information from the German Statistical Agency I expect the percentage of individuals between 16 and 20 years to be around 5 percent (\cite{StatistaWebsite}). Moreover, based on Dutch survey data (\cite{LISSWebsite}) I estimate the share of working individuals in that age range to be around at least 40\%. Based on these estimates I can expect to work with sample sizes of at least 500 individuals per year (I will explain in the following why exactly these parts of the sample are relevant to my analysis). \\

In my empirical analysis I will largely follow Kreiner et al. (\citeyear{KreinerReckMW2020}) in their paper about the Danish MW cutoff at age 18. I will compare young workers between ages 16 and 18 to workers between 18 and 20 using a RD design. Further, I will use this design to estimate the effect of the suddenly applying MW on employment, hours worked and wage rates. Of course, the assumption that workers remain the same before and after their 18\textsuperscript{th} is very arguable. Therefore, I will extend Kreiner's methodology and use data from cohorts that were around the age threshold prior to 2015 as a control group, as then, the MW law hadn't come into effect yet. Effectively I will embed my RD estimates into a DID design where cohorts that turn 18 after 2015 are the treatment group and earlier cohorts the control group. This will ensure that the effects my analysis picks up are not mainly attributable to differences before and after turning 18.
\section{Alternative}
Naturally, there exists the possibility that once I have access to the data, it turns out that my original approach will not be feasible. An example for that would be if the relevant sample sizes are much lower than my current minimum estimates. Should this be the case, I have prepared an alternative approach using which I could study the effects of the German MW law from a different perspective. In this approach I would look at how the MW has affected wage inequality especially in low-income groups. Although MW raise average wages of the lowest earners this does not always mean that wages overall converge. For example, it could be the case that employees do not only care about their own wage, but also about their wage relative to other employees in the company. In that case companies have an incentive to raise wages of all individuals in response to a MW increase (\cite{CardKruegerMWBook2015}).

\printbibliography

\end{document}
